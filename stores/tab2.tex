\begin{longtable}{lc}
\toprule
\textbf{Variable} & \textbf{N = 9,892} \\ 
\midrule
Estrato socioeconómico,  &  \\ 
    Cinco & 167  (1.7\%) \\ 
    Cuatro & 602  (6.1\%) \\ 
    Dos & 4,380  (44\%) \\ 
    Seis & 221  (2.2\%) \\ 
    Tres & 3,483  (35\%) \\ 
    Uno & 1,039  (11\%) \\ 
Sexo,  &  \\ 
    Hombre & 4,973  (50\%) \\ 
    Mujer & 4,919  (50\%) \\ 
Posición dentro del hogar,  &  \\ 
    Descendiente de la cabeza del hogar & 2,322  (23\%) \\ 
    Jefa o jefe del hogar & 4,418  (45\%) \\ 
    Otro & 1,152  (12\%) \\ 
    Pareja de la cabeza del hogar & 2,000  (20\%) \\ 
Nivel educativo más alto alcanzado,  &  \\ 
    Bachillerato & 3,419  (35\%) \\ 
    Ninguno & 46  (0.5\%) \\ 
    Primaria & 1,009  (10\%) \\ 
    Secundaria & 940  (9.5\%) \\ 
    Superior & 4,478  (45\%) \\ 
Ocupación,  &  \\ 
    Empleado doméstico & 563  (5.7\%) \\ 
    Jornalero & 1  (<0.1\%) \\ 
    Obrero del gobierno & 571  (5.8\%) \\ 
    Obrero del sector privado & 8,757  (89\%) \\ 
Empleo formal o informal,  &  \\ 
    Formal & 7,592  (77\%) \\ 
    Informal & 2,300  (23\%) \\ 
Tamaño de la empresa,  &  \\ 
    Empresa pequeña, mediana, o grande & 7,646  (77\%) \\ 
    Microempresa & 2,246  (23\%) \\ 
Edad &  \\ 
    Promedio (Desviación std) & 36 (12) \\ 
    Mínimo y máximo & (18, 86) \\ 
Salario por hora &  \\ 
    Promedio (Desviación std) & 8,822 (12,886) \\ 
    Mínimo y máximo & (327, 350,583) \\ 
\bottomrule
\end{longtable}

